% 
% Annual Cognitive Science Conference
% Sample LaTeX Two-Page Summary -- Proceedings Format
% 

% Original : Ashwin Ram (ashwin@cc.gatech.edu)       04/01/1994
% Modified : Johanna Moore (jmoore@cs.pitt.edu)      03/17/1995
% Modified : David Noelle (noelle@ucsd.edu)          03/15/1996
% Modified : Pat Langley (langley@cs.stanford.edu)   01/26/1997
% Latex2e corrections by Ramin Charles Nakisa        01/28/1997 
% Modified : Tina Eliassi-Rad (eliassi@cs.wisc.edu)  01/31/1998
% Modified : Trisha Yannuzzi (trisha@ircs.upenn.edu) 12/28/1999 (in process)
% Modified : Mary Ellen Foster (M.E.Foster@ed.ac.uk) 12/11/2000
% Modified : Ken Forbus                              01/23/2004
% Modified : Eli M. Silk (esilk@pitt.edu)            05/24/2005
% Modified : Niels Taatgen (taatgen@cmu.edu)         10/24/2006
% Modified : David Noelle (dnoelle@ucmerced.edu)     11/19/2014

%% Change "letterpaper" in the following line to "a4paper" if you must.

\documentclass[10pt,letterpaper]{article}

\usepackage{cogsci}
\usepackage{pslatex}
\usepackage{apacite}
%\usepackage{hyperref}

\title{Tutorial: Bayesian Data Analysis using Probabilistic Programs}
 
\author{{\large \bf Michael Henry Tessler (mhtessler@stanford.edu)}, {\large \bf Noah D. Goodman (ngoodman@stanford.edu)}  \\
  Department of Psychology, Stanford University
  }


\begin{document}

\maketitle

\begin{abstract}

....

\begin{quote}
\small
\textbf{Keywords:} 
bayesian data analysis; bayesian cognitive modeling; probabilistic programming
\end{quote}

\end{abstract}





\section{Significance}

Have you ever collected data and then not known how to analyze it? 
Looked for a difference between conditions but couldn't boil your data down to a single metric? 
The cycle of research has many stages, from experimental design to data collection and writing the results. 
Data analysis is the one stage where young researchers often struggle. 
Analyzing experimental data is an open field of possibilities. 
There is no \emph{one way} to get from data to results. 
Classical tests (t-tests, linear and logistic models) are extremely useful for many cases, but what should you do when the linear model doesn't suffice? 
You could google for your specific question. 
If you're lucky, some statistician or psychologist will have asked your very same question of very similar data and you can plug and play. 
But what if the search turns up nothing?

Bayesian data analysis (BDA) is a general-purpose data analysis approach for making explicit hypotheses about where the data came from (e.g., the hypothesis that data from 2 experimental conditions came from two different distributions). 
This is a fundamentally different approach from the classical statistical framework --- Null Hypothesis Significance Testing (NHST) --- in which the assumptions about how the data were generated are often opaque and force the scientist to adopt assumptions they may or may not be aware of. 
Building the right model for one's data enhances the credibility of the findings and increases the probability of replication. 

%\section{Why is this of interest to CogSci?}

\section{Structure}

This one-day hands-on tutorial will introduce participants to Bayesian Data Analysis, providing a set of tools and techniques that will allow researchers to conduct BDA on their own. 
We will use the probabilistic programming language WebPPL \cite{dippl}, which is freely available to run in the web browser or on the command line, as well as currently via the programming language R. 

\begin{enumerate}
\item Introduction to probabilistic programming
	\subitem Theory: Basics of probability, conditional probability
	\subitem Practical: Introduction to WebPPL, creating arbitrary random variables with programs
\item Simple Bayesian data analysis
	\subitem Theory: Bayes rule 
	\subitem Practical: Posterior inference in WebPPL, comparing models
\item Joint inference models
	\subitem Theory: 
	\subitem Practical:
\item Analyzing cognitive models
	\subitem Theory:
	\subitem Practical:
\item Optimal Experiment Design
	\subitem Theory:
	\subitem Practical:
\end{enumerate}

Each participant will need a laptop but no additional materials are required for this tutorial. 

\section{Organizer Credentials}

Tessler has designed and taught classes on Bayesian Data Analysis and Bayesian cognitive modeling in graduate-level university classes\footnote{
\url{https://web.stanford.edu/class/psych201s/}
} and summer schools around the world\footnote{
e.g., \url{http://stanford.edu/~mtessler/short-courses/2017-computational-pragmatics/}}. 
Tessler has also developed an R package for interfacing with WebPPL \footnote{\url{https://github.com/mhtess/rwebppl}}.
Goodman developed the probabilisitic programming language (PPL) WebPPL, as as well as two other PPLS: Church \cite{church} and Pyro\footnote{\url{http://pyro.ai}}.
He has taught courses on Bayesian cognitive modeling.

\bibliographystyle{apacite}

\setlength{\bibleftmargin}{.125in}
\setlength{\bibindent}{-\bibleftmargin}

\bibliography{tutorial}


\end{document}
