% 
% Annual Cognitive Science Conference
% Sample LaTeX Two-Page Summary -- Proceedings Format
% 

% Original : Ashwin Ram (ashwin@cc.gatech.edu)       04/01/1994
% Modified : Johanna Moore (jmoore@cs.pitt.edu)      03/17/1995
% Modified : David Noelle (noelle@ucsd.edu)          03/15/1996
% Modified : Pat Langley (langley@cs.stanford.edu)   01/26/1997
% Latex2e corrections by Ramin Charles Nakisa        01/28/1997 
% Modified : Tina Eliassi-Rad (eliassi@cs.wisc.edu)  01/31/1998
% Modified : Trisha Yannuzzi (trisha@ircs.upenn.edu) 12/28/1999 (in process)
% Modified : Mary Ellen Foster (M.E.Foster@ed.ac.uk) 12/11/2000
% Modified : Ken Forbus                              01/23/2004
% Modified : Eli M. Silk (esilk@pitt.edu)            05/24/2005
% Modified : Niels Taatgen (taatgen@cmu.edu)         10/24/2006
% Modified : David Noelle (dnoelle@ucmerced.edu)     11/19/2014

%% Change "letterpaper" in the following line to "a4paper" if you must.

\documentclass[10pt,letterpaper]{article}

\usepackage{cogsci}
\usepackage{pslatex}
\usepackage{apacite}
\usepackage{color}

%\usepackage{hyperref}

\newcommand{\ndg}[1]{\textcolor{green}{[ndg: #1]}}

\providecommand{\tightlist}{%
  \setlength{\itemsep}{0pt}\setlength{\parskip}{0pt}}

\title{Statistics as Pottery: Bayesian Data Analysis using Probabilistic Programs \\(Tutorial)}
 
\author{{\large \bf Michael Henry Tessler (mhtessler@stanford.edu)}, {\large \bf Noah D. Goodman (ngoodman@stanford.edu)}  \\
  Department of Psychology, Stanford University
  }


\begin{document}

\maketitle

\begin{abstract}

Drawing inferences from observable data is a fundamental part of science, but how should we do it?
\begin{quote}
\small
\textbf{Keywords:} 
bayesian data analysis; bayesian cognitive modeling; probabilistic programming
\end{quote}

\end{abstract}





\section{Significance}

Learning statistics is like learning pottery. 
With pottery, you can learn how to make different shapes (e.g., a bowl, a vase, a spoon) without understanding general principles. 
Another way is to learn the basic strokes of forming pottery (e.g., how to mold a curved surface, a flat surface, long pointy things). 
In this course, we are going to learn the basic strokes of statistics by building generative models of data. 
We'll compose these strokes to make shapes you've seen before (e.g., a linear model), some shapes you've probably never seen before (\ndg{eg...}), and develop ideas how you would make new shapes if you needed to. 
We won't learn what tests apply to what data types but instead foster the ability to reason through data analysis. 
We will do this through the lens of Bayesian statistics, though the basic ideas will aid your understanding of classical (frequentist) statistics as well.

\ndg{i like the above and below paragraphs, but the transition is a bit jarring.}

%Analyzing experimental data is an open field of possibilities. 
%There is no \emph{one way} to get from data to results. 
%Classical tests (t-tests, linear and logistic models) are extremely useful for many cases, but what should you do when the linear model doesn't suffice? 
%You could google for your specific question. 
%If you're lucky, some statistician or psychologist will have asked your very same question of very similar data and you can plug and play. 
%But what if the search turns up nothing?
Drawing inferences from data is the purview of statistics, and probability theory can be seen as the ``logic of science'' \cite{jaynes2003probability}.
Recent years and commentators have elucidated the shortcomings of the ``classical'' statistical framework --- Null Hypothesis Significance Testing (NHST). 
It is opaque and rife with assumptions the scientist is often forced to adopt.
The inflexibility of NHST is partially credited for the reproducibility crises in psychology and related fields. 

Bayesian data analysis (BDA) is a general-purpose, flexible data analytic approach for drawing inferences about hypotheses from data.
%BDA is fundamentally different from the classical statistical framework --- Null Hypothesis Significance Testing (NHST) --- in which the scientist is often forced to adopt opaque assumptions about how the data were generated, leading to illicit inferences.
%This opacity as well as the inflexibility of NHST can be partially credited for the reproducibility crisis in psychology and related fields. 
BDA provides a principled alternative to NHST: Intuitive conclusions are drawn from intuitive, explicit assumptions.
Major journals have made public their openness to alternatives to NHST \cite<e.g.,>{lindsay2015replication}, and Bayesian techniques are precisely that alternative.

\ndg{ need to say a little more here about what PPLs are, and why they reduce the need for statistical expertise.}
For many years, BDA was a specialist's methodology because it required advanced computational skills to implement and fit Bayesian models.
No longer: With the advent of \emph{probabilistic programming languages} (PPLs) and recent breakthroughs in algorithms for posterior inference, large suites of data analysis problems can be approached from the Bayesian perspective without detailed expertise in statistical algorithms. 
This tutorial will introduce the participant to Bayesian data analysis using probabilistic programs as way to declare models and perform inference. 
By learning how to create simple Bayesian models in a lightweight PPL, participants will more easily grasp the relationship between the \emph{generative process} of the data and the inferences drawn from observed data.
PPLs abstract away nuisance details and allow one to build models from scratch in few lines of code; they further provide black-box inference algorithms that can be used with little expertise.
The PPL approach enhances the transparency of a model, which in turn allows the scientist to reason through the model and revising it. 

\ndg{be a little more explicit here about how doing BDA in a full PPL (not stan) allows BDA of bayesian cognitive models, and why that's good.}
Furthermore, formalizing hypotheses in simple Bayesian models is a first step towards formalizing one's theory in a model. 
A formal model provides a stricter test of one's hypothesis by making explicit the relevant assumptions in one's theory. 
Finally, by formalizing hypotheses and making explicit a set of possible experiments or experimental conditions one could run, one can automate the search for \emph{good experiments} with use of an \emph{optimal experiment design} (OED) system. \ndg{say a little more about OED ad how it's basically free from the PPL BDA setup.}


This tutorial will be of broad interest to the Cognitive Science community because it touches upon a variety of distinct but related topics in the empirical disciplines. 
Foremost, it is a tutorial in modern data analysis and modeling assuming no background knowledge of either Bayesian statistics or probabilisitic programming. 
Rather than specific statistic tests, we teach the basic strokes of statistics (representing the generative process of data by composing probability distributions), which are transferable to whatever domain of inquiry or data the scientist engages with. 
Second, we draw the connection between simple data-analytic models (e.g., regression models) and Bayesian cognitive models, fostering an integrative theoretical view of data analysis and modeling.
%, the analysis of which can both be addressed from the Bayesian data analytic perspective.
Finally, we demonstrate the power of explicit data analysis and modeling by showing how \emph{Optimal Experiment Design} comes for free once a Bayesian data analysis has been chosen.
Throughout, we show participants how to incorporate this kind of modeling into their everyday scientific workflow. 

%\section{Why is this of interest to CogSci?}


\section{Learning Goals}

By the end of the tutorial, participants will be able to 

\begin{enumerate}
\tightlist
\item \textbf{Build} Bayesian statistical models for simple and complex problems using a probabilistic programming language 
\item \textbf{Interpret} the various components of such a model in terms of one's scientific hypotheses 
\item \textbf{Relate} Bayesian model to more orthodox statistical tests (e.g., a linear model) 
\item \textbf{Defend} a particular model specification (priors, likelihood) in a way that other cognitive scientists will understand
\end{enumerate}

\section{Structure}

This one-day hands-on tutorial will introduce participants to Bayesian Data Analysis, providing a set of tools and techniques that will allow researchers to conduct BDA on their own. 
We will use the probabilistic programming language WebPPL \cite{dippl}, which is freely available to run in the web browser or on the command line, as well as via the programming language R.  
\ndg{say we will teach from interactive course notes that will be posted on the web?}

\begin{enumerate}
\tightlist
\item Introduction to probabilistic programming
	\subitem Theory: Basics of probability, conditional probability
	\subitem Introduction to WebPPL / RWebPPL, 
	\subitem Building generative models with programs
\item Simple Bayesian data analysis
	\subitem Theory: Bayes' rule
	\subitem Bayesian inference in WebPPL
	\subitem Highest posterior density intervals
	\subitem Model comparison (e.g., Bayes Factors)
\item Joint inference models
	\subitem Modeling data contamination
	\subitem Building regression models from scratch 
	\subitem Hierarchical modeling  / participant-level parameters
\item Analyzing cognitive models
	\subitem Theory: Cognitive models as model of a hypothesis
	\subitem Learning about the parameters of a cognitive model
\item Optimal Experiment Design
	\subitem Theory: Representing multiple hypotheses as models 
	\subitem Practical: Running \texttt{webppl-oed} on models
\end{enumerate}

Each participant will need a laptop but no additional materials are required for this tutorial. 

\section{Organizer Credentials}

\ndg{revise this. be more explicit about where you taught. begin by saying we are experts in bayesian cognitive modeling and have used BDA extensively in our own research. then say that i've taught bayesian modeling a lot and co-authored probmods which is used all over the place. then say all the great stuff about you having taught bda at a bunch of places.}
Tessler has written tutorials and lectured on Bayesian Data Analysis and Bayesian cognitive modeling in graduate-level university classes\footnote{\url{https://probmods.org/chapters/14-bayesian-data-analysis.html}}
\footnote{\url{http://forestdb.org/models/bayesian-data-analysis.html}}.
In addition, Tessler has designed and taught his own 10-week graduate-level course on Bayesian Data Analysis at Stanford \footnote{
\url{https://web.stanford.edu/class/psych201s/}
} and 1- to 2-week summer schools around the world\footnote{
e.g., \url{http://stanford.edu/~mtessler/short-courses/2017-computational-pragmatics/}}.
Tessler's research integrates Bayesian Data Analysis with Bayesian Cognitive Modeling and has co-authored a paper on Optimal Experiment Design using probabilistic programs \cite{OuyangTLG16}.  
Tessler has also developed an R package for interfacing with WebPPL \footnote{\url{https://github.com/mhtess/rwebppl}}.
Goodman developed the probabilisitic programming language (PPL) WebPPL, as as well as two other PPLS: Church \cite{church} and Pyro\footnote{\url{http://pyro.ai}}.
He teaches courses on probabilistic models of cognition.


\bibliographystyle{apacite}

\setlength{\bibleftmargin}{.125in}
\setlength{\bibindent}{-\bibleftmargin}

\bibliography{tutorial}


\end{document}
